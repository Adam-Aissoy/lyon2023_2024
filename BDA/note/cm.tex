\documentclass{article}

\begin{document}
\section{date 03/10/2023}

%\begin{cm2}
cm lifbda
\subsection{cm2}
12/09/2023 cm2
le modèle relationnel -> 
* page 40 un ensemble est un objet, un panier qui contient des trucs. 
* il y a un tuple dans une relation. pas de double. 
Exemple de tuple: 
soit les relations suivant 
t = <a,b,c,d> 

i leave this txt because i have similary with OneNote.
\subsection{cm3}% absent aysoy 
Cm Adam Aysoy Absent
    -- DF: verifie l'implication  
    --DJ : 

03/10/2023 
%\end{cm2}
\subsection{cm}

%\begin{cmAll}
Cm 
    page 100 
    les dependances d'inclusion ( DI )
    les DI verifie l'inclusion (à verifie ou a demander au prof)
    Q1: c'est quoi les seqences ? Ex de difference entre 
        * R1[AB] inclus R1[AB]
        different de 
        * R1[BA] inclus R1[AB]
        %\alpha + n = n +2
    
    page 101 

Lettre grecque alpha : $\alpha$


%   R1(_ABC_),R1(*(AB)DE)
%    R1[AB] inclus R1[AB]

%Ex: ABCD 
%\Sigma = \sum_{n = 1}^{\infty}

%-> page 102 
%    ** contrainte de 
%-> page 110 
%    ** 
%\Sigma = \sum_{n = 1}^{\infty}  
%\end{cmAll}
\end{document}