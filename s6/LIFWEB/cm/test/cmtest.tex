
\documentclass{article}
\usepackage[utf8]{inputenc}
\usepackage{xcolor} % Pour utiliser \textcolor{}

\title{LIFWEB - CM3 \\ Programmation Fonctionnelle et Asynchrone}
\author{Romuald THION}
\date{Semestre printemps 2023-2024 UCBL}

\begin{document}

\maketitle

\section{PROGRAMMATION FONCTIONNELLE EN JS}
\section{CONCEPTS DE LA PROGRAMMATION FONCTIONNELLE}
Glossary of Modern JavaScript Concepts (2017)
\begin{itemize}
    \item Purity: pure functions, side effects,
    \item State: stateful and stateless,
    \item Immutability and mutability,
    \item Imperative and declarative programming,
    \item Higher-order functions, …
\end{itemize}
Voir LIFPF - Programmation Fonctionnelle.

\section{PROGRAMMATION FONCTIONNELLE AUJOURD’HUI}
2023 Developer Survey Stack Overflow
six langages fonctionnels dans Top paying technologies.
Functional programming is finally going mainstream
\begin{itemize}
    \item “Once functional programming really clicked in my brain, I was like ‘why do it any other way?’”
    \item “React encourages people to think in a more functional way, even if it isn’t purely functional.”
\end{itemize}

\section{FONCTIONS EN JS}
\section{PORTÉE DES VARIABLES}
\textcolor{red}{☠️} var : portée fonction \textcolor{red}{☠️}
\newline
\textcolor{blue}{⚠️} let/const : portée lexicale (bloc) \textcolor{blue}{⚠️}
\newline

\textcolor{green}{👍} let/const, la portée de i est la boucle for.

\begin{verbatim}
for (let i = 0; i < 3; i++) {
  console.log(i); // 0, then 1, then 2
}
%Error: Unmatched {.
\end{verbatim}

\textcolor{orange}{👎} var, la portée de i est la fonction englobante.

\begin{verbatim}
for (var i = 0; i < 3; i++) {
  console.log(i); // 0, then 1, then 2
}
console.log(i); // affiche 3 😱
\end{verbatim}

\section{DÉCLARATION DE FONCTION}
\begin{verbatim}
%Error: Missing \begin{document}.
%Error: Missing \end{document}.
\end{verbatim}

Ici, on crée une fonction avec une instruction (statement) function qui ne produit pas de valeur de retour (MDN).

\section{EXPRESSION FONCTIONNELLE}
\begin{verbatim}
%Error: Missing \begin{document}.
%Error: Missing \end{document}.
\end{verbatim}

👉 Ici, on crée une fonction comme une expression avec laquelle on initialise une variable funct, c’est une Named Function Expression - (NFE) (javascript.info, MDN).

\section{EXPRESSION FLÉCHÉE (1/3)}
A.k.a., arrow functions, fat arrows ou lambdas (MDN).

\begin{verbatim}
%Error: Missing \begin{document}.
%Error: Missing \end{document}.
\end{verbatim}

\textcolor{blue}{⚠️} Sans return, funct renvoie undefined. \textcolor{blue}{⚠️}

\section{EXPRESSION FLÉCHÉE (2/3)}
\begin{verbatim}
%Error: Missing \begin{document}.
%Error: Missing \end{document}.
\end{verbatim}

Quasi-équivalente à la NFE suivante :
\begin{verbatim}
%Error: Missing \begin{document}.
%Error: Missing \end{document}.
\end{verbatim}
Et à l’expression du λ-calcul :

\[λa_1...λa_n.e\]

\section{EXPRESSION FLÉCHÉE (3/3)}
\textcolor{blue}{⚠️} NFE et arrows ne sont pas équivalentes, car une arrow a this dans sa fermeture statique, mais pas la NFE qui l’obtient dynamiquement. \textcolor{blue}{⚠️}

\section{LES FERMETURES (CLOSURES)}
Terme utilisé en JS pour \((λx.e)v\) : l’expression \(e\) dans laquelle \(x\) prend la valeur à \(v\).

\begin{verbatim}
%Error: Missing \begin{document}.
%Error: Missing \end{document}.
\end{verbatim}

💡 helloBuddy() et helloGuys() sont des fermetures (closures) où person est liée.

\section{LES FERMETURES (CLOSURES)}
En λ-calcul (avec \(++\) la concaténation) :

\[helloBuddy ≡ λp.λc.p++c ≡ (helloBuddy) ≡ (λp.λc.p++c)Buddy ≡ (λc.p++c)[p:=Buddy]\]

\section{LA SÉMANTIQUE DE JS}
La sémantique de JS est informelle ECMA-262, sa formalisation se construit a posteriori, e.g. KJS: A Complete Formal Semantics of JavaScript - PLDI’15, voir aussi TypeScript.

\section{EXEMPLE : FERMETURE AVEC LE DOM (1/2)}
\begin{verbatim}
%Error: Missing \begin{document}.
%Error: Missing \end{document}.
\end{verbatim}

💡 La variable taille est liée.

\section{EXEMPLE : FERMETURE AVEC LE DOM (2/2)}
La fermeture de M. Jourdain (TP2) :
\begin{verbatim}
%Error: Missing \begin{document}.
%Error: Missing \end{document}.
\end{verbatim}

\section{EXEMPLE : TABLEAU DE FONCTIONS}
\begin{verbatim}
%Error: Missing \begin{document}.
%Error: Missing \end{document}.
\end{verbatim}

Voir l’exemple.

🤔 let/const n’ont été introduits qu’avec ES6 / ECMAScript 2015, comment faisait-on depuis 1995 ?

\section{IIFE}
An IIFE (Immediately Invoked Function Expression) is a JavaScript function that runs as soon as it is defined MDN.
\begin{verbatim}
%Error: Missing \begin{document}.
%Error: Missing \end{document}.
\end{verbatim}

📜 Permet de limiter la portée des variables avec une closure : méthode avant la standardisation des modules ESM (ES6) pour encapsuler.

💡 Les IIFEs permettent de
