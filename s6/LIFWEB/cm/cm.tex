\documentclass{article}
\usepackage[utf8]{inputenc}
\usepackage{hyperref}
\usepackage[T1]{fontenc}




\title{LIFWEB - CM1}
\author{Oumar Adam ADAM}
\date{Semestre printemps 2023-2024 UCBL}

\begin{document}

\maketitle

\section{Introduction}
Introduction au Web et à JavaScript.

\section{Organisation de l’UE}
\begin{itemize}
    \item Informations générales
    \item Liens emploi du temps
    \item MCC
    \item Programme et supports
\end{itemize}
Site web : \url{http://lifweb.pages.univ-lyon1.fr/}

\section{Doctrine}
\begin{itemize}
    \item Documentation de référence
    \item Pré-requis variés
    \item Use the platform : les standards d’abord
    \begin{itemize}
        \item \url{http://vanilla-js.com/}
        \item \url{https://youmightnotneedjquery.com/}
        \item \url{http://youmightnotneedjs.com/}
    \end{itemize}
    \item Toute la chaîne de la conception au déploiement
    \item Focus sur la qualité, l’automatisation et l’efficacité
\end{itemize}

\section{Faire}
\begin{itemize}
    \item Les exercices 0 sont à préparer avant.
    \item Les TPs sont à finir en intégralité hors BONUS.
    \begin{itemize}
        \item Si vous avez le temps en séance.
        \item Sinon pour approfondir avant les épreuves.
    \end{itemize}
    \item Les assistants LLMs sont autorisés une fois qu’on sait évaluer la qualité de leurs réponses.
\end{itemize}



\section{Programmation Web}
Mousquetaires du Web. HTML5, CSS3, JavaScript et HTTP!

\section{L’écosystème du web}
Les différents métiers du Web.
\begin{itemize}
    \item \url{https://roadmap.sh/frontend}
    \item \url{https://roadmap.sh/backend}
    \item \url{https://roadmap.sh/devops}
    \item \url{https://roadmap.sh/full-stack}
\end{itemize}
Beaucoup de concepts, de technique et d’évolutions.

\section{HTTP}
HTTP status code 413.

\section{Le protocole du Web}
Pile réseau du web.

\section{Basic aspects of HTTP}
\begin{itemize}
    \item HTTP is simple
    \item HTTP is extensible (via headers)
    \item HTTP is stateless, but not sessionless
\end{itemize}

\section{Méthodes des requêtes HTTP}
Appelées aussi verbes.
\begin{itemize}
    \item GET : get a specific resource
    \item POST: create a new resource
    \item PUT: update an existing resource (or create)
    \item DELETE: delete the specified resource
    \item HEAD: get the metadata information
    \item TRACE, OPTIONS, CONNECT, PATCH: avancées
\end{itemize}


\section{Démonstrations}
\begin{itemize}
    \item Devtools (Firefox, Chrome)
    \item \url{https://httpie.io/} et \url{https://curl.se/}
    \item Analyseur de protocole \url{https://www.wireshark.org/}
\end{itemize}

\section{HTML5 et CSS3}
\section{Conseils généraux}
\begin{itemize}
    \item Travaillez systématiquement en navigation privée
    \item Utilisez des sources de référence
    \item Validez vos documents HTML5/CSS3
\end{itemize}


\section{JavaScript : Introduction}

JavaScript est un langage de programmation de haut niveau, principalement utilisé pour développer des applications web interactives et dynamiques. Voici quelques caractéristiques clés de JavaScript :

\begin{itemize}
    \item \textbf{Côté client :} JavaScript est principalement exécuté côté client dans les navigateurs web, ce qui permet d'ajouter des fonctionnalités interactives aux pages web.
    \item \textbf{Polyvalent :} En plus du développement web, JavaScript peut également être utilisé pour développer des applications côté client et côté serveur.
    \item \textbf{Interprété :} JavaScript est un langage interprété, ce qui signifie que le code est exécuté ligne par ligne par un moteur JavaScript intégré au navigateur.
    \item \textbf{Syntaxe inspirée de Java :} Bien que le nom puisse suggérer une relation avec Java, JavaScript est un langage indépendant avec une syntaxe légèrement inspirée de Java.
    \item \textbf{Programmation orientée objet et fonctionnelle :} JavaScript prend en charge les paradigmes de programmation orientée objet et fonctionnelle, offrant une grande flexibilité dans la façon dont le code est structuré.
    \item \textbf{Large écosystème :} JavaScript dispose d'un vaste écosystème de bibliothèques, de frameworks et d'outils qui facilitent le développement web moderne.
\end{itemize}



\subsection{Types de données}
- Cette section traite des différents types de données disponibles en JavaScript, tels que les types primitifs (\texttt{null}, \texttt{undefined}, \texttt{boolean}, \texttt{number}, \texttt{string}, \texttt{symbol}) et les objets.

\subsection{Déclaration de variables}
- Cette section couvre les différentes façons de déclarer des variables en JavaScript, notamment l'utilisation de \texttt{var}, \texttt{let}, et \texttt{const}, ainsi que la portée des variables associées à chaque méthode de déclaration. Par exemple :
\begin{verbatim}
var x = 10;
let y = 20;
const z = 30;
\end{verbatim}

\subsection{Boucles}
- Cette section présente les différentes boucles disponibles en JavaScript, y compris \texttt{for}, \texttt{while}, et \texttt{do...while}, ainsi que les boucles spécifiques comme \texttt{for...in} et \texttt{for...of}. Par exemple :
\begin{verbatim}
for (let i = 0; i < 5; i++) {
    console.log(i);
}
\end{verbatim}

\subsection{Affectation par décomposition}
- Cette section explique le concept d'affectation par décomposition en JavaScript, une fonctionnalité qui permet d'extraire des valeurs d'objets ou de tableaux et de les affecter à des variables distinctes. Par exemple :
\begin{verbatim}
const person = { name: 'John', age: 30 };
const { name, age } = person;
console.log(name, age);
\end{verbatim}

\subsection{JS est impératif / structuré}
- Cette section met en évidence les caractéristiques impératives et structurées de JavaScript, telles que l'affectation de variables, les séquences d'instructions, les boucles et les conditionnelles. Par exemple :
\begin{verbatim}
let x = 10;
if (x > 5) {
    console.log('x is greater than 5');
}
\end{verbatim}

\subsection{JS est orienté objet}
- Cette section aborde la nature orientée objet de JavaScript, en mettant en avant la création d'objets par prototypage et le rôle des fonctions en tant qu'objets de première classe. Par exemple :
\begin{verbatim}
function Person(name) {
    this.name = name;
}
const person = new Person('John');
console.log(person.name);
\end{verbatim}

\subsection{JS est fonctionnel}
- Cette section explore le côté fonctionnel de JavaScript, en mettant l'accent sur les fonctions en tant que valeurs de première classe, les fonctions fléchées, les fermetures lexicales et l'utilisation des fonctions pour les opérations de transformation de données. Par exemple :
\begin{verbatim}
const add = (a, b) => a + b;
console.log(add(2, 3));
\end{verbatim}

\subsection{API fonctionnelle des tableaux}
- Cette section présente l'API fonctionnelle des tableaux en JavaScript, en montrant comment utiliser les méthodes telles que \texttt{map}, \texttt{filter}, \texttt{reduce}, et \texttt{every} pour manipuler les tableaux de manière fonctionnelle. Par exemple :
\begin{verbatim}
const numbers = [1, 2, 3, 4, 5];
const doubled = numbers.map(num => num * 2);
console.log(doubled);
\end{verbatim}

\subsection{JS est événementiel}
- Cette section explique le caractère événementiel de JavaScript, en mettant en évidence la suspension de l'exécution des fonctions, l'adaptation aux longues actions imprévisibles et le modèle de concurrence.


\section{Node.js}
- Cette section présente Node.js, un environnement d'exécution JavaScript côté serveur, en soulignant ses fonctionnalités et son utilisation dans le développement web.

% cm 2 : changement javascript cote client 
% node js 
\subsection {Introduction à Node.js}
Node.js est un environnement d'exécution JavaScript côté serveur qui permet d'exécuter du code JavaScript en dehors d'un navigateur web. Il est basé sur le moteur JavaScript V8 de Google Chrome et offre une grande flexibilité pour le développement d'applications web côté serveur. 
Voici quelques caractéristiques clés de Node.js : 
 









\end{document}
