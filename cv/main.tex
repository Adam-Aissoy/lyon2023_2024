%%%%%%%%%%%%%%%%%
% Ceci est un exemple de CV créé en utilisant altacv.cls (v1.1, 21 novembre 2016) écrit par
% LianTze Lim (liantze@gmail.com), basé sur le
% CV créé par BusinessInsider sur http://www.businessinsider.my/a-sample-resume-for-marissa-mayer-2016-7/?r=US&IR=T
% 
%% Il peut être distribué et/ou modifié selon les
%% conditions de la Licence Publique LaTeX Project, soit la version 1.3
%% de cette licence ou (à votre choix) toute version ultérieure.
%% La dernière version de cette licence est disponible à
%%    http://www.latex-project.org/lppl.txt
%% et la version 1.3 ou ultérieure fait partie de toutes les distributions de LaTeX
%% version 2003/12/01 ou ultérieure.
%%%%%%%%%%%%%%%%

%% Si vous voulez utiliser \orcid ou les
%% icônes académiques, ajoutez "academicons"
%% aux options de \documentclass. 
%% Ensuite, compilez avec XeLaTeX ou LuaLaTeX.
% \documentclass[10pt,a4paper,academicons]{altacv}
\documentclass[10pt,a4paper]{altacv}

%% AltaCV utilise les polices fontawesome et academicon
%% ainsi que les paquets. 
%% Consultez texdoc.net/pkg/fontawecome et http://texdoc.net/pkg/academicons pour la liste complète des symboles.
%% Lors de l'utilisation de l'option "academicons",
%% Compilez avec LuaLaTeX pour de meilleurs résultats. Si vous
%% souhaitez utiliser XeLaTeX, vous devrez peut-être installer
%% Academicons.ttf dans le dossier de polices de votre système d'exploitation.


% Modifiez la mise en page de la page si nécessaire
\geometry{left=1cm,right=9cm,marginparwidth=6.8cm,marginparsep=1.2cm,top=1cm,bottom=1cm}

% Changez la police si vous le souhaitez.

% Si vous utilisez pdflatex :
\usepackage[utf8]{inputenc}
\usepackage[T1]{fontenc}
\usepackage[default]{lato}

% Si vous utilisez xelatex ou lualatex :
% \setmainfont{Lato}

% Changez les couleurs si vous le souhaitez
\definecolor{VividPurple}{HTML}{2E64FE}
\definecolor{SlateGrey}{HTML}{2E2E2E}
\definecolor{LightGrey}{HTML}{666666}
\colorlet{heading}{VividPurple}
\colorlet{accent}{VividPurple}
\colorlet{emphasis}{SlateGrey}
\colorlet{body}{LightGrey}

% Changez les puces pour les listes et le marqueur de notation
% pour \cvskill si vous le souhaitez
\renewcommand{\itemmarker}{{\small\textbullet}}
\renewcommand{\ratingmarker}{\faCircle}

%% sample.bib contient vos publications
\addbibresource{sample.bib}
\usepackage[utf8]{inputenc}% encodage, à modifier selon vos habitudes
\usepackage[scale=0.8]{geometry}% pour régler les marges du CV les options habituelles de l'extension geometry peuvent s'appliquer ici

%% fin ici
\begin{document}
\name{ADAM Oumar Adam}
  \tagline{  }
% Recadrée en carré depuis https://en.wikipedia.org/wiki/Marissa_Mayer#/media/File:Marissa_Mayer_May_2014_(cropped).jpg, CC-BY 2.0
\photo{3cm}{photoAysoy}
\personalinfo{%
  % Toutes ces informations ne sont pas obligatoires !
  % Vous pouvez ajouter les vôtres avec \printinfo{symbole}{détail}
  \email{adamoumaradam26@gmail.com}
  % \phone{+33631779252}
  \phone{+33631779252} 
  
%   \github{} % Je l'invente juste.
%   \orcid{orcid.org/0000-0000-0000-0000} % Bien sûr, je l'invente aussi. Si vous voulez utiliser ce champ (et aussi d'autres symboles académiques), ajoutez l'option "academicons" à \documentclass{altacv}
}

%% Faites en sorte que l'en-tête s'étende jusqu'au bout à droite, si vous le souhaitez. Augmentez la marge droite de 8 cm (=6,8 cm de largeur de marge + 1,2 cm de séparation de marge)
\begin{adjustwidth}{}{-8cm}
\makecvheader
\end{adjustwidth}

%% Fournissez le nom de fichier contenant les contenus de la barre latérale en tant que paramètre facultatif à \cvsection.
%% Vous pouvez toujours utiliser \marginpar{...} si vous ne souhaitez pas aligner le haut des contenus sur un titre \cvsection dans la barre "principale".
Étudiant en licence 3 informatique. Je suis à la recherche d'un emploi étudiant à temps partiel. Je suis disponible les week-ends et tous les soirs avec quelques après-midis.

\cvsection[page1sidebar]{EXPÉRIENCES}
\divider
\cvevent{}{Projet - Université d'Orléans }{Novembre 2020 -- Janvier 2021} {Orléans}
\begin{itemize}
\item Jeu de Nîmes - JAVA
\end{itemize}

\cvevent{}{Projet - Université Claude Bernard Lyon 1 }{ Mars 2022 -- Mai 2022} {Lyon}
\begin{itemize}
\item Jeu de Mini-Golf - C++,SDL2
\end{itemize}

\cvevent{}{Projet - Université Claude Bernard Lyon 1 }{ Mars 2023 -- Mai 2023} {Lyon}
\begin{itemize}
\item Jeu de Course de voiture - C++,SDL2, SFML
\end{itemize}


\divider
\cvevent{}{Responsable de magasin / Dominos Pizza}{Octobre 2021 -- Septembre 2022} {Rillieux-La-Pape, France}


\divider
\cvevent{}{Arbitre de football / FC St Jean le Blanc}{ Janvier 2020- Juillet 2021} {St Jean Le Blanc, France}
\divider


\cvevent{}{Présentateur culturel / Unis Cité}{ Septembre 2019 - Janvier 202


%%%%%%%%%%%%%%% 
% ancien  : 
%%%%%%%%%%%%%%%%%
% This is an example CV created using altacv.cls (v1.1, 21 November 2016) written by
% LianTze Lim (liantze@gmail.com), based on the 
% Cv created by BusinessInsider at http://www.businessinsider.my/a-sample-resume-for-marissa-mayer-2016-7/?r=US&IR=T
% 
%% It may be distributed and/or modified under the
%% conditions of the LaTeX Project Public License, either version 1.3
%% of this license or (at your option) any later version.
%% The latest version of this license is in
%%    http://www.latex-project.org/lppl.txt
%% and version 1.3 or later is part of all distributions of LaTeX
%% version 2003/12/01 or later.
%%%%%%%%%%%%%%%%

%% If you want to use \orcid or the
%% academicons icons, add "academicons"
%% to the \documentclass options. 
%% Then compile with XeLaTeX or LuaLaTeX.
% \documentclass[10pt,a4paper,academicons]{altacv}
\documentclass[10pt,a4paper]{altacv}

%% AltaCV uses the fontawesome and academicon fonts
%% and packages. 
%% See texdoc.net/pkg/fontawecome and http://texdoc.net/pkg/academicons for full list of symbols.
%% When using the "academicons" option,
%% Compile with LuaLaTeX for best results. If you
%% want to use XeLaTeX, you may need to install
%% Academicons.ttf in your operating system's font %% folder.


% Change the page layout if you need to
\geometry{left=1cm,right=9cm,marginparwidth=6.8cm,marginparsep=1.2cm,top=1cm,bottom=1cm}

% Change the font if you want to.

% If using pdflatex:
\usepackage[utf8]{inputenc}
\usepackage[T1]{fontenc}
\usepackage[default]{lato}

% If using xelatex or lualatex:
% \setmainfont{Lato}

% Change the colours if you want to
\definecolor{VividPurple}{HTML}{2E64FE}
\definecolor{SlateGrey}{HTML}{2E2E2E}
\definecolor{LightGrey}{HTML}{666666}
\colorlet{heading}{VividPurple}
\colorlet{accent}{VividPurple}
\colorlet{emphasis}{SlateGrey}
\colorlet{body}{LightGrey}

% Change the bullets for itemize and rating marker
% for \cvskill if you want to
\renewcommand{\itemmarker}{{\small\textbullet}}
\renewcommand{\ratingmarker}{\faCircle}

%% sample.bib contains your publications
\addbibresource{sample.bib}
\usepackage[utf8]{inputenc}% encodage, à modifier selon vos habitudes
\usepackage[scale=0.8]{geometry}% pour régler les marges du CV les options habituelles de l'extension geometry peuvent s'appliquer ici




%% fin ici
\begin{document}
\name{ADAM oumar Adam}
  \tagline{  }
% Cropped to square from https://en.wikipedia.org/wiki/Marissa_Mayer#/media/File:Marissa_Mayer_May_2014_(cropped).jpg, CC-BY 2.0
\photo{3cm}{photoAysoy}
\personalinfo{%
  % Not all of these are required!
  % You can add your own with \printinfo{symbol}{detail}
  \email{adamoumaradam26@gmail.com}
  % \phone{+33631779252}
  \phone{+33631779252} 
  
%   \github{} % I'm just making this up though.
%   \orcid{orcid.org/0000-0000-0000-0000} % Obviously making this up too. If you want to use this field (and also other academicons symbols), add "academicons" option to \documentclass{altacv}
}

%% Make the header extend all the way to the right, if you want. Extend the right margin by 8cm (=6.8cm marginparwidth + 1.2cm marginparsep)
\begin{adjustwidth}{}{-8cm}
\makecvheader
\end{adjustwidth}

%% Provide the file name containing the sidebar contents as an optional parameter to \cvsection.
%% You can always just use \marginpar{...} if you do
%% not need to align the top of the contents to any
%% \cvsection title in the "main" bar.
Etudiant en licence 3 informatique. Je suis à la recherche d’un job étudiant à temps partiel. Je suis disponible les week-ends et tous les soirs avec quelques jours dans l’après midi. 

\cvsection[page1sidebar]{EXPÉRIENCES}
\divider
\cvevent{}{Projet - Université d'Orléans }{Novembre 2020 -- Janvier 2021} {Orléans}
\begin{itemize}
\item Jeu de Nîmes - JAVA
\end{itemize}

\cvevent{}{Projet - Université Claude Bernard Lyon 1 }{ Mars 2022 -- Mai 2022} {Lyon}
\begin{itemize}
\item Jeu de Mini-Golf - C++,SDL2
\end{itemize}

\cvevent{}{Projet - Université Claude Bernard Lyon 1 }{ Mars 2023 -- Mai 2023} {Lyon}
\begin{itemize}
\item Jeu de Course de voiture - C++,SDL2, SFML
\end{itemize}


\divider
\cvevent{}{Responsable de magasin / Dominos Pizza}{Octobre 2021 -- Septembre 2022} {Rillieux-La-Pape, France}


\divider
\cvevent{}{Arbitre de football / FC St Jean le Blanc}{ Janvier 2020- Juillet 2021} {St Jean Le Blanc, France}
\divider


\cvevent{}{Présentateur culturel / Unis Cité}{ Septembre 2019 - Janvier 2020} { Orléans, France}
\begin{itemize}
\item Ciné débat avec les jeunes Lycéens sur des courts et longs
métrages.
\item Contribution à la résolution des problèmes en apportant les
solutions adaptées et en formulant des propositions d'amélioration.
\end{itemize}

\end{document}